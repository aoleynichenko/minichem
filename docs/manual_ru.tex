\documentclass[a4paper]{book}

\usepackage[utf8x]{inputenc}
\usepackage[english,russian]{babel}
\usepackage{indentfirst}
\usepackage{amssymb}
\usepackage{amsfonts}
\usepackage[pdftex]{graphicx}
\usepackage[14pt]{extsizes}
\usepackage[left=2cm,right=2cm,top=3cm,bottom=3cm]{geometry}
\usepackage{hyperref}

\usepackage{titling}
\newcommand{\subtitle}[1]{%
  \posttitle{%
    \par\end{center}
    \begin{center}\large#1\end{center}
    \vskip0.5em}%
}

% programming languages highlighting
\usepackage{listings}
\usepackage{xcolor}
\usepackage{textcomp}
\definecolor{lightlightgray}{gray}{0.9}
\lstset{
	backgroundcolor = \color{lightlightgray},
	basicstyle=\footnotesize\ttfamily
}

\title{Minichem 0.1}
\subtitle{\textit{Руководство пользователя}}
\author{А.\,В. Олейниченко}
\date{22 октября 2016 года}

\begin{document}

\maketitle

\chapter{Общие сведения}
Данный документ является подробным руководством пользователя к квантовохимической программе Minichem версии 0.1.
Программа является учебной, поэтому может содержать ошибки и работать продолжительное время даже на небольших задачах.

\section{Возможности программы}
Доступен расчет энергии в точке методами RHF и UHF.

Отличительной особенностью программы Minichem является полный отказ от традиционных для квантовой химии библиотек линейной алгебры
LAPACK и BLAS и их замена современной высокопроизводительной библиотекой шаблонов Eigen, не только написанной на С++ и обладающей удобным
объектно-ориентированным интерфейсом, но и полностью совместимой с технологией OpenMP. По производительности Eigen не уступает MKL
(\url{http://eigen.tuxfamily.org/index.php?title=Benchmark}).

Для расчета интегралов на атомных базисных функциях гауссова типа используется библиотека LIBINT \cite{libint2}. Несмотря на интеграцию с LIBINT,
сохраняется возможность вызова <<родных>> для Minichem подпрограмм расчета интегралов, которые не отличаются высокой производительностью.

\chapter{Установка и запуск}
\section{Установка}
Программа Minichem устанавливается из исходных кодов, доступных на GitHub по адресу \url{https://github.com/aoleynichenko/minichem}.
Для сборки требуются установленные библиотеки LIBINT, Eigen и OpenMP. Детали их установки зависят от дистрибутива и здесь не приводятся.
На текущий момент мейкфайлы написаны с учетом специфики ОС Linux.
%\lstset{language=bash}
Команды для сборки Minichem:

\begin{lstlisting}
cd \$MINICHEM_HOME/src
make
make install
\end{lstlisting}
Исполняемый файл Minichem появится в \texttt{\$MINICHEM\_HOME/bin}.

\section{Запуск}
Minichem -- консольное приложение. Формат команды запуска:
\begin{lstlisting}
minichem <options> <input-files>
\end{lstlisting}
Удобно добавить директорию \texttt{\$MINICHEM\_HOME/bin} в \texttt{\$PATH}.

По умолчанию вывод результатов осуществляется в стандартный поток вывода stdout.

\section{Диагностика ошибок. Лог}
Кроме основного вывода, Minichem также ведет лог
(по умолчанию -- minichem.log), в который записываются сообщения о ходе исполнения программы. Лог содержит развернутые сообщения о произошедших
ошибках и предупреждениях и может оказаться полезным при отладке как самого Minichem, так и input-скриптов.

\chapter{Описание формата входных файлов}
\section{Язык Minichem}
Для входных файлов Minichem разрабатывается специальный скриптовый язык программирования, ориентированный на применение в вычислительной химии
и наследующий черты языков Psithon, NWChem и JavaScript.

Язык Minichem чувствителен к регистру (кроме названий базисных наборов и символов химических элементов). Символы конца строки и лишние пробельные символы
игрнорируются, что в принципе позволяет записать весь входной файл в одну строчку. Это возможно благодаря тому, что каждый оператор языка <<знает>>, сколько
аргументов должен принять.

Комментарии начинаются с символа \texttt{\#} и продолжаются до конца строки (аналогично Python и Bash).

\section{Типы данных}
\textbf{Molecule.} Объекты типа Molecule содержат информацию о заряде молекулы, мультиплетности основного состояния и координатах атомов.
Для создания объекта Molecule используется ключевое слово \texttt{mol}:
\begin{lstlisting}
mol LiH {
  mult 1
  charge 0
  units atomic
  H    0.0000   0.0000   0.0000
  Li   1.4632   0.0000   0.0000
}
\end{lstlisting}
Обязательными в определении молекулы являются только ключевое слово \texttt{mol} и координаты атомов в формате xyz.
Параметры молекулы, указываемые ключевыми словами mult, charge, units и т.д. необязательны.
После ключевого слова \texttt{mol} опционально указывается имя молекулы; если оно не указано, объявленная
молекула становится текущей (доступной для квантовохимических подпрограмм). Пример объявления безымянной молекулы:
\begin{lstlisting}
# methane molecule
mol {
  C    0.0000   0.0000   0.0000
  H    0.6281   0.6281  -0.6281
  H   -0.6281  -0.6281  -0.6281
  H    0.6281  -0.6281   0.6281
  H   -0.6281   0.6281   0.6281
}
\end{lstlisting}

Вместо символов элементов можно указывать заряды ядер.

\textbf{Basis Set.} Используемый в расчетах базис атомных орбиталей представляет собой набор гауссовых функций, центрированных на атомах;
технически это динамический массив \texttt{std::vector}, состоящий из объектов типа \texttt{libint2::Shell}.
Набор АО для каждого атома генерируется по <<шаблону>>, который не совсем точно может быть назван \textit{базисным набором} (basis set).
Задать базисный набор можно с помощью ключевого слова \texttt{basis}:
\begin{lstlisting}
basis cc-pVTZ
\end{lstlisting}
Данная директива <<заставляет>> Minichem генерировать АО из базисного набора cc-pVTZ на \textit{всех} атомах заданной молекулы.
Тот же смысл имеет конструкция
\begin{lstlisting}
basis {
  * library cc-pVTZ
}
\end{lstlisting}
Можно задавать базисные наборы для \textit{каждого} типа атомов:
\begin{lstlisting}
basis mixed_bs {
  O library sto-3g
  H library 6-31g**
}
\end{lstlisting}
Названия библиотечных базисных наборов, указываемые после ключевого слова \texttt{library}, регистронезависимы и могут содержать любые символы
(знаки дефиса, звездочки и т.д.).
Наконец, можно задать базисный набор вручную (экспоненты записываются в левой колонке, каждая следующая колонка отвечает коэффициентам
сжатой базисной функции):
\begin{lstlisting}
basis custom-basis-HeH {
  spherical
  H library cc-pvtz
  # actually, it is hand-written cc-pVTZ basis set for He
  He    S
    234.0000000      0.0025870
     35.1600000      0.0195330
      7.9890000      0.0909980
      2.2120000      0.2720500
  He    S
      0.6669000      1.0000000
  He    S
      0.2089000      1.0000000
  He    P
      3.0440000      1.0000000
  He    P
      0.7580000      1.0000000
  He    D
      1.9650000      1.0000000
}
\end{lstlisting}
Как очевидно из последнего примера, задание базисных наборов вручную может соседствовать с директивами \texttt{basis}. Это может оказаться
удобным для расширения <<стандартного>> базисного набора диффузными функциями или использования базиса, отсутствующего в библиотеке Minichem.

По умолчанию Minichem генирирует базис сферических гауссовых функций, <<включить>> декартов базис можно указанием ключевого слова \texttt{cartesian}:
\begin{lstlisting}
basis cart-3-21g {
  cartesian
  N    S
      242.7660000      0.0598657
       36.4851000      0.3529550
        7.8144900      0.7065130
  N    SP
        5.4252200     -0.4133010      0.2379720
        1.1491500      1.2244200      0.8589530
  N    SP
        0.2832050      1.0000000      1.0000000
}
\end{lstlisting}

Ссылки на оригинальные публикации приведены в начале каждого библиотечного файла с базисным набором.
Стиль оформления базисных наборов позаимствован у программного пакета NWChem.

\begin{thebibliography}{9}
\bibitem{libint2} LIBINT: A library for the evaluation of molecular integrals of many-body operators over Gaussian functions, Version 2.1.0 (beta)
Edward F. Valeev, http://libint.valeyev.net/ .
\end{thebibliography}

\end{document}
