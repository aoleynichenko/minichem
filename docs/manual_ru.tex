\documentclass[a4paper]{book}

\usepackage[utf8x]{inputenc}
\usepackage[english,russian]{babel}
\usepackage{indentfirst}
\usepackage{amssymb}
\usepackage{amsfonts}
\usepackage[pdftex]{graphicx}
\usepackage[14pt]{extsizes}
\usepackage[left=2cm,right=2cm,top=3cm,bottom=3cm]{geometry}
\usepackage{hyperref}

\usepackage{titling}
\newcommand{\subtitle}[1]{%
  \posttitle{%
    \par\end{center}
    \begin{center}\large#1\end{center}
    \vskip0.5em}%
}

% programming languages highlighting
\usepackage{listings}
\usepackage{xcolor}
\usepackage{textcomp}
\definecolor{lightlightgray}{gray}{0.9}
\lstset{
	backgroundcolor = \color{lightlightgray},
	basicstyle=\footnotesize\ttfamily
}

\title{Minichem 0.1}
\subtitle{\textit{Руководство пользователя}}
\author{А.\,В. Олейниченко}
\date{18 октября 2016 года}

\begin{document}

\maketitle

\chapter{Общие сведения}
Данный документ является подробным руководством пользователя к квантовохимической программе Minichem версии 0.1.
Программа является учебной, поэтому может содержать ошибки и работать продолжительное время даже на небольших задачах.

\section{Возможности программы}
Доступен расчет энергии в точке методами RHF и UHF.

\chapter{Установка и запуск}
\section{Установка}
Программа Minichem устанавливается из исходных кодов, доступных на GitHub по адресу \url{https://github.com/aoleynichenko/minichem}.
Для сборки требуются установленные библиотеки LAPACK, BLAS, OpenMP. Детали их установки зависят от дистрибутива и здесь не приводятся.
На текущий момент мейкфайлы написаны с учетом специфики ОС Linux.
%\lstset{language=bash}
Команды для сборки Minichem:

\begin{lstlisting}
cd \$MINICHEM_HOME/src
make
make install
\end{lstlisting}
Исполняемый файл Minichem появится в \texttt{\$MINICHEM\_HOME/bin}.

\section{Запуск}
Minichem -- консольное приложение. Формат команды запуска:
\begin{lstlisting}
minichem <options> <input-files>
\end{lstlisting}
Удобно добавить директорию \texttt{\$MINICHEM\_HOME/bin} в \texttt{\$PATH}.

По умолчанию вывод результатов осуществляется в стандартный поток вывода stdout.

\section{Диагностика ошибок. Лог}
Кроме основного вывода, Minichem также ведет лог
(по умолчанию -- minichem.log), в который записываются сообщения о ходе исполнения программы. Лог содержит развернутые сообщения о произошедших
ошибках и предупреждениях и может оказаться полезным при отладке как самого Minichem, так и input-скриптов.

\chapter{Описание формата входных файлов}
\section{Язык Minichem}
Для входных файлов Minichem разрабатывается специальный скриптовый язык программирования, ориентированный на применение в вычислительной химии
и наследующий черты языков Psithon, NWChem и JavaScript.

Язык Minichem чувствителен к регистру (кроме названий базисных наборов и символов химических элементов). Символы конца строки и лишние пробельные символы
игрнорируются, что в принципе позволяет записать весь входной файл в одну строчку. Это возможно благодаря тому, что каждый оператор языка <<знает>>, сколько
аргументов должен принять.

\section{Типы данных}
\textbf{Molecule.} Объекты типа Molecule содержат информацию о заряде молекулы, мультиплетности основного состояния и координатах атомов.
Для создания объекта Molecule используется ключевое слово \texttt{mol}:
\begin{lstlisting}
mol LiH {
  mult=1
  charge=0
  units=atomic
  H    0.0000   0.0000   0.0000
  Li   1.4632   0.0000   0.0000
}
\end{lstlisting}
Обязательными в определении молекулы являются только ключевое слово \texttt{mol} и координаты атомов в формате xyz.
Параметры, указываемые через знак равенства (mult, charge, units...) необязательны.
После ключевого слова \texttt{mol} опционально указывается имя молекулы; если оно не указано, объявленная
молекула становится текущей (доступной для квантовохимических подпрограмм). Пример объявления безымянной молекулы:
\begin{lstlisting}
mol {
  C    0.0000   0.0000   0.0000
  H    0.6281   0.6281  -0.6281
  H   -0.6281  -0.6281  -0.6281
  H    0.6281  -0.6281   0.6281
  H   -0.6281   0.6281   0.6281
}
\end{lstlisting}

\end{document}



