\documentclass[a4paper, 12pt]{article}

%\usepackage[a4paper, includefoot,
%			left=2.5cm,
%			right=1.5cm,
%			top=2cm,
%			bottom=2cm,			
%			headsep=1cm,
%			footskip=1cm]{geometry}


\usepackage[T1]{fontenc}
\usepackage[utf8x]{inputenc}
\usepackage[english,russian]{babel}
\usepackage{indentfirst}
\usepackage{amssymb}
\usepackage{amsfonts}
\usepackage{amsmath}
\usepackage{hyperref}
\usepackage[pdftex]{graphicx}
\usepackage[14pt]{extsizes}
\usepackage[left=2cm,right=2cm,top=3cm,bottom=3cm]{geometry}
\usepackage{hyperref}
\usepackage{tabularx}
\usepackage{float}
\usepackage{bold-extra}
\usepackage{titling}
\usepackage{graphicx}
\usepackage{braket}
\usepackage{bold-extra}   % bold TT font

% for left eigenvectors
\newcommand{\Perp}{{\perp\perp}}

% for keywords description lists
\usepackage{scrextend}
\addtokomafont{labelinglabel}{\ttfamily}

\newcommand{\subtitle}[1]{%
  \posttitle{%
    \par\end{center}
    \begin{center}\large#1\end{center}
    \vskip0.5em}%
}


% programming languages highlighting
\usepackage{listings}
\usepackage{xcolor}
\usepackage{textcomp}
\definecolor{lightlightgray}{gray}{0.93}
\lstset{
	backgroundcolor = \color{lightlightgray},
	basicstyle=\small\ttfamily
}

% for italic captions
\usepackage[format=plain,
            labelfont={bf,it},
            textfont=it]{caption}

% advanced tables
\usepackage{tabularx}
\newcolumntype{L}{>{\arraybackslash}m{10cm}}



%\bibliographystyle{}

\title{{\bf minichem} \\ \bigskip \normalsize Версия 1.0 }
\author{Александр Олейниченко}
\subtitle{\textit{Руководство пользователя}}

\begin{document}

\maketitle

\tableofcontents

\section{Общие сведения}

\texttt{minichem} -- это небольшая учебная квантовохимическая программа, разработанная с целью поближе познакомиться с техническими аспектами реализации различных методов решения электронного уравнения Шредингера. В ходе работы над программой автор опирался как на учебные пособия, так и на оригинальные работы и (к сожалению, немногочисленные) веб-ресурсы (см. список литературы). Среди использованных материалов два учебных пособия представляются наиболее полно отражающими сложности и особенности алгоритмической реализации методов квантовой химии, а потому особенно важными:

\begin{itemize}
\item \cite{szabo1996} A. Szabo, N. Ostlund, "Modern Quantim Chemistry";
\item \cite{helgaker2008} T. Helgaker, P. Jorgensen, J. Olsen, "Molecular Electronic-Structure Theory".
\end{itemize}

Исходный текст программы полностью написан на языке программирования С (стандарт C99). В настоящий момент программа ориентирована на Unix-подобные операционные системы; тем не менее, в будущем программа может быть портирована и на другие платформы.

Ниже перечислены основные возможности программы, предоставляемые пользователю:

\begin{itemize}
\item расчет энергии системы (атома или молекулы) при заданной геометрии;
\item ограниченный (RHF) и неограниченный (UHF) методы Хартри-Фока-Рутаана;
\item техника DIIS улучшения сходимости процедуры решения уравнений ССП;
\item распараллеливание на основе технологии OpenMP;
\item доступные химические элементы: вся периодическая система (внимание: программа никак не учитывает релятивистские эффекты!);
\item базисные наборы: декартовы, с произвольным угловым моментом базисных функций.
\end{itemize}

Исходный код проекта доступен на Github:

\url{https://github.com/aoleynichenko/minichem}

\bigskip

Автор будет рад любым замечаниям, вопросам и предложениям:

\href{mailto:alexvoleynichenko@gmail.com}{\nolinkurl{alexvoleynichenko@gmail.com}}.


\section{Компиляция и тестирование программы}

Перед началом сборки необходимо убедиться, что на Вашей машине установлены следующие библиотеки:

\begin{itemize}
\item \texttt{libc} -- стандартная библиотека языка C;
\item \texttt{MPI} -- библиотека для параллельных вычислений для систем с распределенной памятью;
\item \texttt{OpenMP} -- библиотека для параллельных вычислений для систем с общей памятью;
\item \texttt{BLAS} и \texttt{LAPACK} -- библиотеки линейной алгебры.
\end{itemize}

Для непосредственно сборки потребуются инструменты \texttt{CMake} и \texttt{make}, а также компиляторы языка C (рекомендованы компиляторы GNU или Intel).

Для компиляции зайдите в домашний каталог \texttt{minichem} и выполните следующие команды:

\begin{lstlisting}
$ mkdir build && cd build
$ cmake ..
$ make [-jN]
\end{lstlisting}

В случае успешной компиляции в каталоге \texttt{build} появится исполняемый файл \texttt{minichem.x}.

Тесты размещены в папке \texttt{test}. Система тестирования написана на языке Python2 (должен быть предустановлен). Для тестирования выполните команды:

\begin{lstlisting}
$ cd test
$ python test.py
\end{lstlisting}


\section{Структура входных файлов и запуск задач}

Запуск программы \texttt{minichem}:
\begin{lstlisting}
$ minichem.x <input-file.inp>
\end{lstlisting}

Программа направляет результаты работы в стандартный поток вывода, поэтому удобнее сразу перенаправлять его в файл командой \texttt{tee}:
\begin{lstlisting}
$ minichem.x <input-file.inp> | tee <output-file.out>
\end{lstlisting}

Дизайн языка входных файлов позаимствован у программы NWChem \cite{nwchem2010}. Так же как и NWChem, \texttt{minichem} работает как интерпретатор, <<исполняя>> входной файл как программу, написанную на очень простом скриптовом языке. Язык \texttt{minichem} во многом совпадает с языком NWChem; везде, где это было возможно, названия директив и секций были выбраны так, чтобы входной файл мог быть интерпретирован обоими квантовохимическими программами без дополнительных правок.

Входные файлы \texttt{minichem} строятся из директив и секций. Директивы -- это однострочные команды, секции объединяют наборы директив. Как правило, секции задают параметры для работы какого-либо модуля (например, параметры сходимости процедуры ССП). \texttt{minichem} последовательно считывает строки входного файла и соответственно заданным в нем параметрам изменяет свои внутренние переменные. Как только \texttt{minichem} доходит до директивы \texttt{task}, исполнение входного файла останавливается и начинается расчет. В одном входном файле допустимы несколько директив \texttt{task}; между ними пользователь может менять параметры расчетов, организовывая, таким образом, каскад вычислений.

Язык \texttt{minichem} также поддерживает однострочные комментарии, начинающиеся со знака \texttt{'\#'}. Язык нечувствителен к регистру.

В качестве примера входного файла приведем файл, составленный для молекулы бензола в базисе STO-3G. Пояснения ко всем директивам содержатся в комментариях.
Подробные описания всех директив и секций даны в разделе \ref{sec:details}. Несколько других примеров можно найти в разделе \ref{sec:examples}.

\begin{lstlisting}
# C6H6 single-point energy
#
# RHF molecular orbitals will be written to the molden-format
# file c6h6.mos

# name for the task
start C6H6

# allowed memory usage
memory 10 mb

# print input file before executing it
echo

# number of OpenMP threads
nproc 8

# geometry (cartesian)
# default: charge = 0
geometry
  C    0.000    1.396    0.000
  C    1.209    0.698    0.000
  C    1.209   -0.698    0.000
  C    0.000   -1.396    0.000
  C   -1.209   -0.698    0.000
  C   -1.209    0.698    0.000
  H    0.000    2.479    0.000
  H    2.147    1.240    0.000
  H    2.147   -1.240    0.000
  H    0.000   -2.479    0.000
  H   -2.147   -1.240    0.000
  H   -2.147    1.240    0.000
end

# basis set specification
# (keyword SPHERICAL -- only for compatibility with NWChem)
basis "ao basis" SPHERICAL
H    S
      3.42525091             0.15432897       
      0.62391373             0.53532814       
      0.16885540             0.44463454  
C    S
     71.6168370              0.15432897       
     13.0450960              0.53532814       
      3.5305122              0.44463454       
C    S
      2.9412494             -0.09996723
      0.6834831              0.39951283    
      0.2222899              0.70011547            
C    P
      2.9412494              0.15591627       
      0.6834831              0.60768372       
      0.2222899              0.39195739    
end

# options for the SCF module
# by default: singlet
scf
  print "overlap"   # print AO overlap integrals
  diis 5            # enable DIIS, max subspace dim = 5
  maxiter 20        # max number of SCF iterations
end

# export calculated data (molecular orbitals)
out
  molden   # to the MOLDEN .mos format
end

# do RHF calculation
task scf
\end{lstlisting}


\section{Директивы входных файлов}\label{sec:details}

\subsection{start}

Синтаксис:

\begin{lstlisting}
start <string name>
\end{lstlisting}

Директива \texttt{start} задает небольшой не содержащий пробелов идентификатор задачи. Этот идентификатор используется затем в названиях временных и выходных файлов.

\subsection{echo}

Синтаксис:

\begin{lstlisting}
echo
\end{lstlisting}

Если директива \texttt{echo} встречается во входном файле, \texttt{minichem} направляет содержимое входного файла в стандартный вывод. Рекомендуется всегда использовать эту директиву.

\subsection{memory}

Синтаксис:

\begin{lstlisting}
memory <integer> <string units>
# <units>: one of b (bytes), kb, mb, mw (megawords), gb
\end{lstlisting}

Максимальное количество оперативной памяти, которое может быть выделено и использовано программой.

\subsection{nproc}

Синтаксис:

\begin{lstlisting}
nproc <integer>
\end{lstlisting}

Число OpenMP-нитей (для параллельного исполнения).

\subsection{task}

Синтаксис:

\begin{lstlisting}
task <string theory>
\end{lstlisting}

Директива задает квантовохимический метод, который должен быть использован для решения электронной задачи.

Поскольку в настоящее время в программе реализован только метод Хартри-Фока, директива \texttt{task} может вызываться только с аргументом \texttt{scf}:
\begin{lstlisting}
task scf
\end{lstlisting}

\subsection{geometry}

Синтаксис:

\begin{lstlisting}
geometry [units <string units default angstroms>]
  [charge <integer default 0>]
  [mult <integer default 1>]
  <string elem> <real x y z>
  . . .
end
\end{lstlisting}

Составная директива \texttt{geometry} задает декартовы координаты входящих в молекулу атомов, единицы измерения расстояний (бор или ангстрем), заряд молекулы и спиновую мультиплетность. По заданным значениям заряда и мультиплетности программа может автоматически выбрать, какой вариант метода Хартри-Фока необходимо использовать.

\subsection{charge}
Синтаксис:

\begin{lstlisting}
charge <integer>
\end{lstlisting}

Директива задает общий заряд системы (в атомных единицах). Добавлена для совместимости с NWChem.

\subsection{basis}

Синтаксис:

\begin{lstlisting}
basis [<string name default "ao basis">] [spherical|cartesian]
  <string elem> <string shell_L>
    <real exponent> <real list_of_coefficients>
    . . .
  . . .
end
\end{lstlisting}

Описание базисного набора. В настоящее время реализованы только декартовы базисные наборы (ключевое слово \texttt{cartesian}), поэтому ключевое слово \texttt{spherical} сейчас необходимо только для совместимости с NWChem.

Коэффициенты сжатых функций расположены в колонках; первая колонка -- показатели экспонент гауссовых примитивов.

Угловой момент блока базисных функций (\texttt{<string shell\_L>}) обозначается, как обычно, буквами \texttt{S}, \texttt{P}, \texttt{D} ... Можно использовать сокращение \texttt{SP} (одна сжатая функция типа $s$, вторая -- типа $p$). \texttt{minichem} не ставит ограничений на максимальный угловой момент базисных функций.

\subsection{scf}

Синтаксис:

\begin{lstlisting}
scf
  [rhf | uhf]
  [singlet | doublet | triplet | quartet | quintet]
  [guess (core | eht)]
  [direct | nodirect]
  [maxiter <integer max_no_of_iterations default 50>]
  [diis [<integer subspace_dim default 5>] | nodiis]
  [print <string what>]
  [noprint <string what>]
end
\end{lstlisting}

Составная директива \texttt{scf} используется для задания параметров процедуры решения уравнений ССП. Может содержать одну или несколько директив из следующего списка:

\begin{labeling}{alligator}
\item [guess] начальное приближение к МО. Возможные значения аргумента:
	\begin{itemize}
	\item \texttt{core} -- приближение голых ядер: полное пренебрежение двухэлектронными интегралами. Хорошо работает только для систем с небольшим числом электронов;
	\item \texttt{eht} -- расширенный метод Хюккеля \cite{hoffmann1963} (в варианте Вольфс\-берга-Гельмгольца \cite{wolfsberg1952}).
	\end{itemize}
    По умолчанию -- расширенный метод Хюккеля.
\item [rhf/uhf] вариант метода Хартри-Фока (\texttt{rhf} -- ограниченный, \texttt{uhf} -- неограниченный).
\item [singlet/doublet/triplet/quartet/quintet] спиновая мультиплетность системы.
\item [maxiter] максимальное число итераций решения уравнений ССП. Значение по умолчанию: 50.
\item [diis] использовать технику DIIS \cite{pulay1980,pulay1981} для ускорения сходимости уравнений ССП. Необязательный аргумент директивы -- максимальная размерность пространства итераций, которые используются для экстраполяции (значение по умолчанию: 5). Использование DIIS может быть рекомендовано практически во всех случаях и поэтому включено по умолчанию.
\item [nodiis] не использовать технику DIIS.
\item [direct] использовать <<прямой>> вариант метода ССП (двухэлектронные интегралы рассчитываются на ходу на каждой итерации, а не считываются из предварительно подготовленного файла). <<Прямой>> ССП может работать на порядок медленнее <<обычного>>. Иногда может быть полезен в случае нехватки дискового пространства.
\item [nodirect] использовать <<обычный>> вариант метода ССП (хранение интегралов на диске) (по умолчанию).
\item [print] какая дополнительная информация должна быть напечатана (но не печатается по умолчанию). Тип аргумента -- строка в двойных кавычках. Возможные аргументы директивы \texttt{print} приведены в таблице:

{
\small
\begin{tabular}{|l|l|}
\hline
\texttt{"final vectors analysis"} & коэффициенты разложения МО \\
\texttt{"overlap"}   & интегралы перекрывания в АО-базисе \\
\texttt{"kinetic"}   & интегралы оператора кинетической энергии \\
                     & энергии \\
\texttt{"potential"} & интегралы оператора электрон-ядерной \\
                     & потенциальной энергии \\
\texttt{"eri"}       & двухэлектронные кулоновские интегралы \\
\hline
\end{tabular}
}
\item [noprint] работает прямо противоположно директиве \texttt{print} (запрещает печать дополнительной информации). Аргументы такие же, как у \texttt{print} (см. выше).
\end{labeling}

\subsection{out}

Синтаксис:

\begin{lstlisting}
out
  [molden]
end
\end{lstlisting}

Составная директива \texttt{out} позволяет экспортировать результаты расчетов для дальнейшей их обработки другими программами. Может содержать следующие директивы:

\begin{labeling}{alligator}
\item [molden] экспортировать информацию о базисном наборе и молекулярные орбитали в файл формата MOLDEN (\texttt{.mos}) \cite{molden2000,molden2017,MoldenFormat}. В настоящий момент экспорт реализован только для метода RHF.
\end{labeling}

\section{Примеры входных файлов для типовых задач}\label{sec:examples}

UHF-расчет атома лития в базисе STO-3G:
\begin{lstlisting}
start Li
echo

geometry units atomic
  Li 0 0 0
end

basis "ao basis" SPHERICAL
Li    S
     16.1195750              0.15432897       
      2.9362007              0.53532814       
      0.7946505              0.44463454       
Li    SP
      0.6362897             -0.09996723             0.15591627       
      0.1478601              0.39951283             0.60768372       
      0.0480887              0.70011547             0.39195739       
end

scf
  uhf
  diis
  doublet
end

task scf
\end{lstlisting}

\bigskip

UHF-расчет дублетной частицы $\rm CH_3$:
\begin{lstlisting}
start CH3
echo

geometry
C     0.08745162    -0.08744725    -0.08742186
H     0.52503428     0.78909888    -0.52508604
H    -0.78884450    -0.52530342    -0.52536318
H     0.52530257    -0.52529218     0.78892712
end

# STO-3G
basis "ao basis" SPHERICAL
H    S
      3.42525091             0.15432897       
      0.62391373             0.53532814       
     0.16885540             0.44463454  
C    S
     71.6168370              0.15432897       
     13.0450960              0.53532814       
      3.5305122              0.44463454       
C    S
      2.9412494             -0.09996723
      0.6834831              0.39951283    
      0.2222899              0.70011547            
C    P
      2.9412494              0.15591627       
      0.6834831              0.60768372       
      0.2222899              0.39195739    
end

scf
  uhf
  doublet
end

task scf
\end{lstlisting}

\clearpage

\bibliographystyle{unsrt}
\bibliography{manual-refs}{}

\end{document}
